\documentclass[a4paper,12pt]{article}
\usepackage{listings}
\lstset{language=Lisp}
\newcommand{\subpar}[1]{\medskip \noindent #1.}

\begin{document}

\subpar{a}

\begin{lstlisting}
((lambda (n)
   ((lambda (fact)
      (fact fact n))
    (lambda (ft k)
      (if (= k 1)
          1
          (* k (ft ft (- k 1)))))))
 10)

((lambda (fact)
   (fact fact 10))
 (lambda (ft k)
   (if (= k 1)
       1
       (* k (ft ft (- k 1))))))

((lambda (ft k)
   (if (= k 1)
       1
       (* k (ft ft (- k 1)))))
 (lambda (ft k)
   (if (= k 1)
       1
       (* k (ft ft (- k 1)))))
 10)

(* 10
   ((lambda (ft k)
      (if (= k 1)
          1
          (* k (ft ft (- k 1)))))
    (lambda (ft k)
      (if (= k 1)
          1
          (* k (ft ft (- k 1)))))
    9))

(* 10
   (* 9
      ((lambda (ft k)
         (if (= k 1)
             1
             (* k (ft ft (- k 1)))))
       (lambda (ft k)
         (if (= k 1)
             1
             (* k (ft ft (- k 1)))))
       8)))
\end{lstlisting}

Here's an analogous expression for computing Fibonacci numbers.

\begin{lstlisting}
(lambda (n)
  ((lambda (fib)
     (fib fib n))
   (lambda (ft k)
     (if (or (= k 1) (= k 2))
         1
         (+ (ft ft (- k 1))
            (ft ft (- k 2)))))))
\end{lstlisting}

\subpar{b}
\begin{lstlisting}
(define (f x)
  ((lambda (even? odd?)
     (even? even? odd? x))
   (lambda (ev? od? n)
     (if (= n 0) #t (od? ev? od? (- n 1))))
   (lambda (ev? od? n)
     (if (= n 0) #f (ev? ev? od? (- n 1))))))
\end{lstlisting}
\end{document}
