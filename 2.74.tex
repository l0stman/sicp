\documentclass[a4paper,12pt]{article}
\usepackage{listings}
\newcommand{\subpar}[1] {\medskip \noindent #1.} \lstset{language=Lisp}

\begin{document}
\subpar{a}
\begin{lstlisting}
(define (get-record name file)
  ((get 'record (type-tag file)) name))
\end{lstlisting}
Each personnel file should be tagged with the division name.  Each
division file should implements a record operator, keyed under the
division name, that returns the record of an employee in that
division.

\subpar{b}
\begin{lstlisting}
(define (get-salary record)
  ((get 'salary (type-tag record)) record))
\end{lstlisting}
A record should be tagged with the division name to differentiate
them.  And an operator salary should be keyed under the division name.

\subpar{c}
\begin{lstlisting}
(define (find-employee-record name file-list)
  (if (null? file-list)
      (error "Can't find record for employee -- " name)
      (let ((record (get-record name (car file-list))))
        (if record
            record
            (find-employee-record name
                                  (cdr file-list))))))
\end{lstlisting}

\subpar{d}  There's no change to be made to the existing records.  The
new personnel information just needs to implement the previous
operators under the new company's name.
\end{document}
