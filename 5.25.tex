\documentclass[a4paper,12pt]{article}
\usepackage{listings}
\lstset{language=Lisp}
\newcommand{\subpar}[1]{\medskip \noindent #1.}

\begin{document}

First, let's define the \lstinline!ACTUAL-VALUE! procedure which is
used when we need an expression's value instead of just calling
\lstinline!EVAL-DISPATCH!.

\begin{lstlisting}
ACTUAL-VALUE
(save continue)
(assign continue (label FORCE-IT))
(goto (label EVAL-DISPATCH))
FORCE-IT
(test (op thunk?) (reg val))
(branch (label FORCE-IT-THUNK))
(test (op evaluated-thunk?) (reg val))
(branch (label THUNK-VALUE))
(restore continue)
(goto (reg continue))
FORCE-IT-THUNK
(save val)
(assign exp (op thunk-exp) (reg val))
(assign env (op thunk-env) (reg val))
(assign continue (label FORCE-IT-GOT-VALUE))
(goto (label ACTUAL-VALUE))
FORCE-IT-GOT-VALUE
(restore exp)                    ; object to force
(perform (op set-car!)
         (reg exp)
         (const evaluated-thunk))
(assign exp (op cdr) (reg exp))
(perform (op set-car!) (reg exp) (reg val))
(perform (op set-cdr!) (reg exp) (reg ()))
(restore continue)
(goto (reg continue))
THUNK-VALUE
(assign val (op thunk-value) (reg val))
(restore continue)
(goto (reg continue))
\end{lstlisting}

We also define the \lstinline!DELAY-IT! procedure for abstraction
purpose and to produce shorter code when evaluating an application
expression.

\begin{lstlisting}
DELAY-IT
(assign val (op delay-it) (reg exp) (reg env))
(goto (reg continue))
\end{lstlisting}

Then we define \lstinline!EV-APPLICATION! which evaluates the operator
and dispatches to the correct procedure type, doing nothing with the
operands.

\begin{lstlisting}
EV-APPLICATION
(save continue)
(save env)
(assign unev (op operands) (reg exp))
(save unev)
(assign exp (op operator) (reg exp))
(assign continue (label EV-APPL-DID-OPERATOR))
(goto (label ACTUAL-VALUE))
EV-APPL-DID-OPERATOR
(restore unev)                          ; the operands
(restore env)
(assign proc (reg val))                 ; the operator
(test (op primitive-procedure?) (reg proc))
(branch (label PRIMITIVE-APPLY))
(test (op compound-procedure?) (reg proc))
(branch (label COMPOUND-APPLY))
(goto (label UNKNOWN-PROCEDURE-TYPE))
\end{lstlisting}

For a primitive procedure, we evaluate all the arguments before
applying the primitive

\begin{lstlisting}
PRIMITIVE-APPLY
(assign continue (label PRIM-APPLY-TO-ARGS))
(assign exp (label ACTUAL-VALUE))
(goto (label EV-APPL-OPERANDS))
PRIM-APPLY-TO-ARGS
(assign val (op apply-primitive-procedure)
            (reg proc)
            (reg argl))
(restore continue)
(goto (reg continue))
\end{lstlisting}

For a compound procedure, we delay all the arguments before applying
the procedure

\begin{lstlisting}
COMPOUND-APPLY
(assign continue (label COMP-APPL-TO-ARGS))
(assign exp (label DELAY-IT))
(goto (label EV-APPL-OPERANDS))
COMP-APPL-TO-ARGS
(assign unev (op procedure-parameters) (reg proc))
(assign env (op procedure-environment) (reg proc))
(assign env (op extend-environment)
            (reg unev)
            (reg argl)
            (reg env))
(assign unev (op procedure-body) (reg proc))
(goto (label EV-SEQUENCE))
\end{lstlisting}

When evaluating the operands, we store into the register \textbf{argl}
a list which is the result of applying to each element of the operands
the procedure stored in the register \textbf{exp}.

\begin{lstlisting}
EV-APPL-OPERANDS
(assign argl (op empty-arglist))
(test (op no-operands?) (reg unev))
(branch (label EV-APPL-NO-ARGS))
(save continue)
(save proc)
(save exp)
EV-APPL-OPERAND-LOOP
(restore proc) ; procedure to apply to the operand
(assign exp (op first-operand) (reg unev))
(test (op last-operand?) (reg unev))
(branch (label EV-APPL-LAST-ARG))
(save proc)
(save argl)
(save env)
(save unev)
(assign continue (label EV-APPL-ACCUMULATE-ARG))
(goto (reg proc))
EV-APPL-ACCUMULATE-ARG
(restore unev)
(restore env)
(restore argl)
(assign argl (op adjoin-arg) (reg val) (reg argl))
(assign unev (op rest-operands) (reg unev))
(goto (label EV-APPL-OPERAND-LOOP))
EV-APPL-LAST-ARG
(save argl)
(assign continue (label EV-APPL-ACCUM-LAST-ARG))
(goto (label proc))
EV-APPL-ACCUM-LAST-ARG
(restore argl)
(assign argl (op adjoin-arg) (reg val) (reg argl))
(restore proc)
(restore continue)
EV-APPL-NO-ARGS
(goto (reg continue))
\end{lstlisting}

We modify \lstinline!EV-IF! to use \lstinline!ACTUAL-VALUE! when
evaluating the predicate.

\begin{lstlisting}
EV-IF
(save exp) ; save expression for later
(save env)
(save continue)
(assign continue (label EV-IF-DECIDE))
(assign exp (op if-predicate) (reg exp))
(goto (label ACTUAL-VALUE)) ; evaluate the predicate
EV-IF-DECIDE
(restore continue)
(restore env)
(restore exp)
(test (op true?) (reg val))
(branch (label EV-IF-CONSEQUENT))
EV-IF-ALTERNATIVE
(assign exp (op if-alternative) (reg exp))
(goto (label EVAL-DISPATCH))
EV-IF-CONSEQUENT
(assign exp (op if-consequent) (reg exp))
(goto (label EVAL-DISPATCH))
\end{lstlisting}

And finally, we change the \lstinline!READ-EVAL-PRINT-LOOP! procedure
to also use \lstinline!ACTUAL-VALUE!.

\begin{lstlisting}
READ-EVAL-PRINT-LOOP
(perform (op initialize-stack))
(perform (op prompt-for-input)
         (const ";;; LEC-EVAL input:"))
(assign exp (op read))
(assign env (op get-global-environment))
(assign continue (label PRINT-RESULT))
(goto (label ACTUAL-VALUE))
PRINT-RESULT
(perform (op announce-output)
         (const ";;; LEC-Eval value:"))
(perform (op user-print) (reg val))
(goto (label READ-EVAL-PRINT-LOOP))
\end{lstlisting}

\end{document}
