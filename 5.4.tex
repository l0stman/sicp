\documentclass[a4paper,12pt]{article}
\newcommand{\subpar}[1]{\medskip \noindent #1.}
\newcommand{\la}{\leftarrow}
\usepackage{listings}
\usepackage{pgf}
\usepgfmodule{plot,shapes,matrix}
\usepackage{tikz}
\usetikzlibrary{matrix,positioning,shapes,arrows,calc}

\makeatletter
\pgfdeclareshape{forbidden parking}
{
  \inheritsavedanchors[from=circle] % this is nearly a circle
  \inheritanchorborder[from=circle]
  \inheritanchor[from=circle]{north}
  \inheritanchor[from=circle]{north west}
  \inheritanchor[from=circle]{north east}
  \inheritanchor[from=circle]{center}
  \inheritanchor[from=circle]{west}
  \inheritanchor[from=circle]{east}
  \inheritanchor[from=circle]{mid}
  \inheritanchor[from=circle]{mid west}
  \inheritanchor[from=circle]{mid east}
  \inheritanchor[from=circle]{base}
  \inheritanchor[from=circle]{base west}
  \inheritanchor[from=circle]{base east}
  \inheritanchor[from=circle]{south}
  \inheritanchor[from=circle]{south west}
  \inheritanchor[from=circle]{south east}
  \inheritbackgroundpath[from=circle]
  \foregroundpath{
    \centerpoint%
    \pgf@xc=\pgf@x%
    \pgf@yc=\pgf@y%
    \pgfutil@tempdima=\radius%
    \pgfmathsetlength{\pgf@xb}{\pgfkeysvalueof{/pgf/outer xsep}}%
    \pgfmathsetlength{\pgf@yb}{\pgfkeysvalueof{/pgf/outer ysep}}%
    \ifdim\pgf@xb<\pgf@yb%
      \advance\pgfutil@tempdima by-\pgf@yb%
    \else%
      \advance\pgfutil@tempdima by-\pgf@xb%
    \fi%
    \pgfpathmoveto{\pgfpointadd{\pgfqpoint{\pgf@xc}{\pgf@yc}}{\pgfqpoint{-0.707107\pgfutil@tempdima}{-0.707107\pgfutil@tempdima}}}
    \pgfpathlineto{\pgfpointadd{\pgfqpoint{\pgf@xc}{\pgf@yc}}{\pgfqpoint{0.707107\pgfutil@tempdima}{0.707107\pgfutil@tempdima}}}
    \pgfpathmoveto{\pgfpointadd{\pgfqpoint{\pgf@xc}{\pgf@yc}}{\pgfpoint{-0.707107\pgfutil@tempdima}{0.707107\pgfutil@tempdima}}}
    \pgfpathlineto{\pgfpointadd{\pgfqpoint{\pgf@xc}{\pgf@yc}}{\pgfqpoint{0.707107\pgfutil@tempdima}{-0.707107\pgfutil@tempdima}}}
  }
}
\makeatother

\newenvironment{datapath}
               {
                 \begin{tikzpicture}
                   [
                     >=stealth',
                     constant/.style={
                       isosceles triangle,
                       isosceles triangle apex angle=60,
                       draw,
                       font=\tiny
                     },
                     register/.style={
                       rectangle,
                       minimum width=15mm,
                       draw,
                       text height=1.1ex,
                       text depth=.20ex,
                     },
                     button/.style={
                       forbidden parking,
                       inner sep=3pt,
                       draw
                     },
                     operation/.style={
                       trapezium,
                       draw,
                       text height=1.1ex,
                       text depth=.20ex,
                     },
                     test/.style={
                       circle,
                       draw
                     },
                     point/.style={
                       circle,
                       fill,
                       draw,
                       inner sep=1.5pt
                     },
                     every label/.style={font=\scriptsize}
                   ]
               }
               {\end{tikzpicture}}

\newenvironment{controller}
               {
                 \begin{tikzpicture}
                   [
                     font=\scriptsize,
                     button/.style={rectangle,draw},
                     test/.style={diamond,draw},
                   ]
               }
               {\end{tikzpicture}}

\begin{document}

\subpar{a} For the recursive exponentiation, we use the following data
paths diagram

\begin{datapath}
  \matrix[row sep=5mm,column sep=5mm] {
    &&& \node[constant,shape border rotate=-90] (zero) {$0$}; \\
    \node[constant,shape border rotate=-90] (one) {$1$}; &
    \node[operation] (minus) {$-$}; &
    \node[register] (n) {n}; &
    \node[test] (equal) {$=$}; \\
    \node[button,label=left:val$\la$1] (val init) {}; &&
    \node[register] (stack) {stack}; \\
    \node[register] (val) {val}; \\
    \node[register] (b) {b}; & &
    \node[register] (continue) {continue}; &
    \node[font=\scriptsize] (controller) {controller}; \\
  };

  \node[button,label=above:n$\la$n$-$1] (decrement n) at
  ($(minus)!.5!(n) + (0,.75)$) {};
  \node[point] (one inter south) at ($(one.south)!.5!(val
  init.north)$) {};
  \node[operation,shape border rotate=270] (mul) at
       ($(val)!.5!(b)+ (1.5,0)$) {$*$};
  \node[point] (val inter north) at ($(val init.south)!.4!(val.north)$) {};
  \node[button,label=above:val$\la$val$*$b] (update val) at (val inter
  north -| mul) {};
  \node[button,label=left:rc] (restore) at
  ($(continue.north)!.5!(stack.south) - (.4,0)$) {};
  \node[button,label=right:sc] (save) at
  ($(continue.north)!.5!(stack.south) + (.4,0)$) {};
  \node[button,below=of continue -| restore,label=left:c$\la$after-expt]
  (after exp cont) {};
  \node[button,below=of continue -| save,label=right:c$\la$expt-done]
  (exp done cont) {};
  \node[constant,label=left:after-expt,shape border rotate=90] (after exp) at
       ($(after exp cont.south) - (0,.75)$) {};
  \node[constant,label=right:expt-done,shape border rotate=90] (exp done) at
       ($(exp done cont.south) - (0,.75)$) {};

  \draw (minus) |- (decrement n) [->] -| (n);
  \draw (n) [->] -- (equal);
  \draw (zero) [->] -- (equal);
  \draw (one) -- (val init) [->] -- (val);
  \draw (n) -- ($(n.south)!.5!(stack.north)$) [->] -| (minus.south
  east);
  \draw (one inter south) [->] -| (minus.south west);
  \draw [->] (val.south) --
  ($(val.south)!.18!(b.north)$) node (val inflexion south) {} --
  (val inflexion south -| mul.west);
  \draw [->] (b.north) --
  ($(b.north)!.18!(val.south)$) node (b inflexion north) {} --
  (b inflexion north -| mul.west);
  \draw (mul.east) -- ++(.25,0) |- (update val) -- (val inter north);
  \draw (continue) [->] -- (controller);
  \draw (stack.south -| restore) -- (restore) [->] --
  (restore |- continue.north);
  \draw (save |- continue.north) -- (save) [->] -- (stack.south -|
  save);
  \draw (after exp) -- (after exp cont) [->] -- (after exp cont |-
  continue.south);
  \draw (exp done) -- (exp done cont) [->] -- (exp done cont |-
  continue.south);
\end{datapath}

\begin{lstlisting}
(controller
   (assign continue (label expt-done))
 expt-loop
   (test (op =) (reg n) (const 0))
   (branch (label base-case))
   (save continue)
   (assign continue (label after-expt))
   (assign n (op -) (reg n) (const 1))
   (goto (label expt-loop))
 after-expt
   (restore continue)
   (assign val (op *) (reg val) (reg b))
   (goto (reg continue))
 base-case
   (assign val (const 1))
   (goto (reg continue))
 expt-done)
\end{lstlisting}

\subpar{b} The iterative exponentiation doesn't need a stack since the
recursive call is in tail call position.

\begin{datapath}
  \matrix[row sep=5mm,column sep=5mm] {
    & \node[register] (n) {n}; \\
    & \node[button,label=left:c$\la$n] (counter init) {}; &
    \node[constant,shape border rotate=-90] (zero) {$0$}; \\
    \node[constant,shape border rotate=-90] (one) {$1$}; &
    \node[register] (counter) {counter}; &
    \node[test] (equal) {=}; \\
    &\node[button,label=right:c$\la$c$-$1] (decrement counter) {}; \\
    \node[button,label=left:p$\la$1] (product init) {}; &
    \node[register] (product) {product}; \\
    & \node[register] (b) {b}; \\
  };

  \node [operation,shape border rotate=180] (minus) at
  ($(one.east)!.5!(counter.west) - (0,1.25)$) {$-$};
  \node [operation,shape border rotate=270] (mul) at
  ($(product.south)!.5!(b.north) + (1.5,0)$) {$*$};
  \node [point] (one inter south) at ($(one.south) + (0,-.2)$) {};
  \node [button,label=right:p$\la$p$*$b] (update product) at
  ($(mul.east) + (.1,1)$) {};

  \draw[->] (n) -- (counter init) -- (counter);
  \draw[->] (counter) -- (equal);
  \draw[->] (zero) -- (equal);
  \draw (one) -- (product init) [->] -- (product);
  \draw[->] (counter.west) -- ++(-.1,0) node (inflexion right) {} --
  (inflexion right |- minus.north);
  \draw[->] (one inter south) -- (one inter south -| one.east) -- ++
  (.1,0) node (inflexion left) {} -- (inflexion left |- minus.north);
  \draw[->] (minus.south) -- ++(0,-.1) -| (decrement counter) --
  (counter);
  \draw[->] (product.south) --
  ($(product.south)!.18!(b.north)$) node (product inflexion south) {} --
  (product inflexion south -| mul.west);
  \draw[->] (b.north) --
  ($(b.north)!.18!(product.south)$) node (b inflexion north) {} --
  (b inflexion north -| mul.west);
  \draw[->] (mul.east) -| (update product) -| (product.north);
\end{datapath}

\begin{lstlisting}
(controller
   (assign counter (reg n))
   (assign product (const 1))
 test-counter
   (test (op =) (reg counter) (const 0))
   (branch (label expt-done))
   (assign counter (op -) (reg counter) (const 1))
   (assign product (op *) (reg product) (reg b))
   (goto (label test-counter))
 expt-done)
\end{lstlisting}

\end{document}
