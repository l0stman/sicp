\documentclass[a4paper,12pt]{article}
\usepackage{listings}
\lstset{language=Lisp}
\newcommand{\subpar}[1]{\medskip \noindent #1.}

\begin{document}

Let's define the factorial machine as follows

\begin{lstlisting}
(define fact-machine
  (make-machine
   (list
    (list 'initialize
          (lambda ()
            ((fact-machine 'stack) 'initialize)))
    (list 'print-statistics
          (lambda ()
            ((fact-machine 'stack) 'print-statistics)))
    (list '= =)
    (list '- -)
    (list '* *)
    (list 'read read))
   '(FACT-LOOP
     (perform (op initialize))
     (assign n (op read))
     (assign continue (label FACT-DONE))
     TEST-N
     (test (op =) (reg n) (const 1))
     (branch (label BASE-CASE))
     (save continue)
     (save n)
     (assign n (op -) (reg n) (const 1))
     (assign continue (label AFTER-FACT))
     (goto (label TEST-N))
     AFTER-FACT
     (restore n)
     (restore continue)
     (assign val (op *) (reg n) (reg val))
     (goto (reg continue))
     BASE-CASE
     (assign val (const 1))
     (goto (reg continue))
     FACT-DONE
     (perform (op print-statistics))
     (goto (label FACT-LOOP)))))
\end{lstlisting}

Here are the results we obtain for small values of $n$.

\begin{lstlisting}
> (start fact-machine)
1
(total-pushes = 0 maximum-depth = 0)
2
(total-pushes = 2 maximum-depth = 2)
3
(total-pushes = 4 maximum-depth = 4)
4
(total-pushes = 6 maximum-depth = 6)
5
(total-pushes = 8 maximum-depth = 8)
6
(total-pushes = 10 maximum-depth = 10)
7
(total-pushes = 12 maximum-depth = 12)
8
(total-pushes = 14 maximum-depth = 14)
9
(total-pushes = 16 maximum-depth = 16)
10
(total-pushes = 18 maximum-depth = 18)
\end{lstlisting}

From these numbers, we could deduce that the maximum stack depth and
the number of push operations to compute $n!$ is $2(n-1)$.  And we can
verify easily by induction that it's the case for $n\ge1$.

\end{document}
