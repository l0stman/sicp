\documentclass[a4paper,12pt]{article}
\usepackage{listings}
\lstset{language=Lisp}
\newcommand{\subpar}[1]{\medskip \noindent #1.}

\begin{document}

\begin{lstlisting}
;;; EC-EVAL input:
(define (fib n)
  (if (< n 2)
      n
      (+ (fib (- n 1))
         (fib (- n 2)))))
(total-pushes = 3 maximum-depth = 3)

;;; EC-Eval value:
ok

;;; EC-EVAL input:
(fib 0)
(total-pushes = 16 maximum-depth = 8)

;;; EC-Eval value:
0

;;; EC-EVAL input:
(fib 1)
(total-pushes = 16 maximum-depth = 8)

;;; EC-Eval value:
1

;;; EC-EVAL input:
(fib 2)
(total-pushes = 72 maximum-depth = 13)

;;; EC-Eval value:
1

;;; EC-EVAL input:
(fib 3)
(total-pushes = 128 maximum-depth = 18)

;;; EC-Eval value:
2

;;; EC-EVAL input:
(fib 4)
(total-pushes = 240 maximum-depth = 23)

;;; EC-Eval value:
3

;;; EC-EVAL input:
(fib 5)
(total-pushes = 408 maximum-depth = 28)

;;; EC-Eval value:
5
\end{lstlisting}

\subpar{a} We then deduce that the maximum depth of the stack required
to compute $\mathrm{Fib}(n)$ is $5 n + 3$ for $n \ge 2$.

\subpar{b} Let $S(n)$ be the number of pushes used in computing
$\mathrm{Fib}(n)$.  If we note $k$ the number pushes
produced by the call to $<$, $+$ and the two calls to $-$ then we have
easily for $n \ge 2$

\[ S(n) = S(n-1) + S(n-2) + k.\]

Moreover, if we note $c$ the number of pushes of a call to $<$, then
\[ S(0) = S(1) = c.\]

From these two properties, We could easily show by induction that for
$n \ge 0$
\[ S(n) = (c+k) \mathrm{Fib}(n+1) - k.\]

From the results of the simulation, we have $c = 16$ and  $k = 40$,
hence we have for $n \ge 0$
\[ S(n) = 56\ \mathrm{Fib}(n+1) - 40.\]

We then deduce that the number of pushes to compute $\mathrm{Fib}(n)$
grows exponentially with $n$.
\end{document}
