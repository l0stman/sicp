\documentclass[a4paper,12pt]{article}
\usepackage{listings}
\lstset{language=Lisp}
\newcommand{\subpar}[1]{\medskip \noindent #1.}

\begin{document}

We just add the following clause to the definition of \lstinline!eval!

\begin{lstlisting}
  ((unless? exp) (eval-unless exp env))
\end{lstlisting}

And then we define \lstinline!eval-unless! as follow

\begin{lstlisting}
(define (eval-unless exp env)
  (if (true? (eval (if-predicate exp) env))
      (eval (if-alternative exp) env)
      (eval (if-consequent exp) env)))
\end{lstlisting}

Now suppose we have three lists of scheme expressions of the same
length.  We want to return a list whose $n^{th}$ is the $n^{th}$
element of the second list if the evaluation of the $n^{th}$ element
of the first list is true, or the $n^{th}$ element of the third list
otherwise.  If \lstinline!unless! is implemented as function, we could
just use

\begin{lstlisting}
  (map unless list1 list2 list3)
\end{lstlisting}

instead of doing explicitly

\begin{lstlisting}
(define (loop lst1 lst2 lst3)
  (if (null? list1)
      '()
      (cons (unless (car lst1) (car lst2) (car lst3))
            (loop (cdr lst1) (cdr lst2) (cdr lst3)))))
\end{lstlisting}

since we see that there's already a high-order function which could do
the job fine.
\end{document}
