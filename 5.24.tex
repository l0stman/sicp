\documentclass[a4paper,12pt]{article}
\usepackage{listings}
\lstset{language=Lisp}
\newcommand{\subpar}[1]{\medskip \noindent #1.}

\begin{document}

First we define the following operations and we add them to
\lstinline!eceval-operations!
\begin{lstlisting}
(define (clauses exp) (cdr exp))
(define (empty-clauses? clauses) (null? clauses))
(define (first-clause clauses) (car clauses))
(define (rest-clauses clauses) (cdr clauses))
(define (else? exp) (tagged-list? exp 'else))
(define (clause-predicate cl) (car cl))
(define (clause-actions cl) (cdr cl))
\end{lstlisting}

Then we replace the former definition of EV-COND with the following
code:

\begin{lstlisting}
EV-COND
(assign exp (op clauses) (reg exp))
EV-COND-LOOP
(test (op empty-clauses?) (reg exp))
(branch (label EV-COND-DONE))
(assign unev (op first-clause) (reg exp))
(assign exp (op rest-clauses) (reg exp))
(test (op else?) (reg unev))
(branch (label EV-COND-EXEC-CLAUSE))
(save exp)
(save env)
(save continue)
(assign continue (label EV-COND-DECIDE))
(assign exp (op clause-predicate) (reg unev))
(save unev)
(goto (label EVAL-DISPATCH))
EV-COND-DECIDE
(restore unev)                          ; current clause
(restore continue)
(restore env)
(restore exp)                           ; rest of clauses
(test (op true?) (reg val))
(branch (label EV-COND-EXEC-CLAUSE))
(goto (label EV-COND-LOOP))
EV-COND-EXEC-CLAUSE
(save continue)
(assign unev (op clause-actions) (reg unev))
(goto (label EV-SEQUENCE))
EV-COND-DONE
(goto (reg continue))
\end{lstlisting}

\end{document}
