\documentclass[a4paper,12pt]{article}
\usepackage{listings}
\lstset{language=Lisp}

\begin{document}
Let's consider two intervals $I_1 = [l_1, u_1]$ and $ I_2 = [l_2,
  u_2]$.  The sum of the interval is,
\[ S = [l_1+l_2, u_1+u_2].\]
Thus,
\begin{eqnarray*}
\mathrm{width}(S) &=& \frac{(u_1+u_2) - (l_1+l_2)}{2} \\
&=& \frac{u_1-l_1}{2} + \frac{u_2-l_2}{2} \\
&=& \mathrm{width}(I_1) + \mathrm{width}(I_2)
\end{eqnarray*}
The difference $D$ between the two intervals $I_1$ and $I_2$ is the
sum of the interval $I_1$ and $-I_2$.  Thus we have,
\begin{eqnarray*}
\mathrm{width}(D) &=& \mathrm{width}(I_1) + \mathrm{width}(-I_2) \\
&=& \mathrm{width}(I_1) + \mathrm{width}(I_2)
\end{eqnarray*}
Let's show that for multiplication, this property doesn't hold.
\begin{eqnarray*}
I_1 &=& [0, 1],\ I_2 = [2, 3],\ P = [0, 3],\ \mathrm{width}(P) = 3 \\
I_1 &=& [1/2, 3/2],\ I_2 = [2, 3],\ P = [1, 9/2],\ \mathrm{width}(P) =
7/2
\end{eqnarray*}
We see that even if the width of the interval $I_1$ is always 1, the
product's width varies.  We could do the same reasonning for division.
\end{document}
