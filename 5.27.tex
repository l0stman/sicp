\documentclass[a4paper,12pt]{article}
\usepackage{listings}
\lstset{language=Lisp}
\newcommand{\subpar}[1]{\medskip \noindent #1.}

\begin{document}

We obtain the following simulation for the recursive factorial.
\begin{lstlisting}
;;; EC-EVAL input:
(define (factorial n)
  (if (= n 1)
      1
      (* (factorial (- n 1)) n)))
(total-pushes = 3 maximum-depth = 3)

;;; EC-Eval value:
ok

;;; EC-EVAL input:
(factorial 1)
(total-pushes = 16 maximum-depth = 8)

;;; EC-Eval value:
1

;;; EC-EVAL input:
(factorial 2)
(total-pushes = 48 maximum-depth = 13)

;;; EC-Eval value:
2

;;; EC-EVAL input:
(factorial 3)
(total-pushes = 80 maximum-depth = 18)

;;; EC-Eval value:
6

;;; EC-EVAL input:
(factorial 4)
(total-pushes = 112 maximum-depth = 23)

;;; EC-Eval value:
24

;;; EC-EVAL input:
(factorial 5)
(total-pushes = 144 maximum-depth = 28)

;;; EC-Eval value:
120

;;; EC-EVAL input:
(factorial 6)
(total-pushes = 176 maximum-depth = 33)

;;; EC-Eval value:
720
\end{lstlisting}

We then deduce from these numbers the following table
\begin{quote}
  \begin{tabular}{|c|c|c|}
    \hline &Maximum depth& Number of pushes \\
    \hline Recursive factorial& $5 n + 3$ & $32 n - 16$ \\
    \hline Iterative factorial& $10$ & $35 n + 26$ \\
    \hline
  \end{tabular}
\end{quote}

\end{document}
