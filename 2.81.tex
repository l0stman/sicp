\documentclass[a4paper,12pt]{article}
\usepackage{listings}
\newcommand{\subpar}[1] {\medskip \noindent #1.}
\lstset{language=Lisp}

\begin{document}
\subpar{a} If $x$ and $y$ are of type \lstinline!scheme-number!,
there's no coercion going on.  \lstinline!expt! is applied to the
\lstinline!contents! of $x$ and $y$.

On the other hand if we call \lstinline!exp! with two complex numbers,
there's no \lstinline!exp! operation for the type
\lstinline!(complex complex)!.  Thus, $x$ is coerced into a complex
type, which is $x$ itself and the expression
\lstinline!(apply-generic exp x y)! is evaluated.  So we get an
infinite loop.

\subpar{b}  There's no need to do coercion with arguments of the same
type.  If we're doing coercion, the operation doesn't exists for the
type of arguments.  But a coercion with same type would just leave the
arguments unchanged.  So the current algorithm handles just fine the
case.

\subpar{c}
\begin{lstlisting}
(define (apply-generic op . args)
  (let ((type-tags (map type-tag args)))
    (let ((proc (get op type-tags)))
      (if proc
          (apply proc (map contents args))
          (if (and (= (length args) 2)
                   (eq? (car type-tags) (cadr type-tags)))
              (let ((type1 (car type-tags))
                    (type2 (cadr type-tags))
                    (a1 (car args))
                    (a2 (cadr args)))
                (let ((t1->t2 (get-coercion type1 type2))
                      (t2->t1 (get-coercion type2 type1)))
                  (cond (t1->t2
                         (apply-generic op (t1->t2 a1) a2))
                        (t2->t1
                         (apply-generic op a1 (t2->t1 a2)))
                        (else
                         (error "No method for these types"
                                (list op type-tags))))))
              (error "No method for these types"
                     (list op type-tags)))))))
\end{lstlisting}
\end{document}
