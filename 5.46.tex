\documentclass[a4paper,12pt]{article}
\usepackage{listings}
\lstset{language=Lisp}
\newcommand{\subpar}[1]{\medskip \noindent #1.}

\begin{document}

Here is some samples obtained by simulating the interpreted Fibonacci
program.

\begin{lstlisting}
;;; EC-EVAL input:
(define (fib n)
  (if (< n 2)
      n
      (+ (fib (- n 1)) (fib (- n 2)))))
(total-pushes = 3 maximum-depth = 3)

;;; EC-Eval value:
ok

;;; EC-EVAL input:
(fib 0)
(total-pushes = 12 maximum-depth = 8)

;;; EC-Eval value:
0

;;; EC-EVAL input:
(fib 1)
(total-pushes = 12 maximum-depth = 8)

;;; EC-Eval value:
1

;;; EC-EVAL input:
(fib 2)
(total-pushes = 54 maximum-depth = 13)

;;; EC-Eval value:
1

;;; EC-EVAL input:
(fib 3)
(total-pushes = 96 maximum-depth = 18)

;;; EC-Eval value:
2

;;; EC-EVAL input:
(fib 4)
(total-pushes = 180 maximum-depth = 23)

;;; EC-Eval value:
3

;;; EC-EVAL input:
(fib 5)
(total-pushes = 306 maximum-depth = 28)

;;; EC-Eval value:
5
\end{lstlisting}

Using the same reasoning as in \textbf{exercise 5.29}, we deduce that
the number of pushes to compute $\mathrm{Fib}(n)$ is $42
\mathrm{Fib(n+1)}-30$ and the maximum depth is $5n + 3$ for $n \ge 2$.

For the compiled code we obtain

\begin{lstlisting}
> (compile-and-go
   '(define (fib n)
      (if (< n 2)
          n
          (+ (fib (- n 1)) (fib (- n 2))))))
(total-pushes = 0 maximum-depth = 0)

;;; EC-Eval value:
ok

;;; EC-EVAL input:
(fib 0)
(total-pushes = 5 maximum-depth = 3)

;;; EC-Eval value:
0

;;; EC-EVAL input:
(fib 1)
(total-pushes = 5 maximum-depth = 3)

;;; EC-Eval value:
1

;;; EC-EVAL input:
(fib 2)
(total-pushes = 13 maximum-depth = 4)

;;; EC-Eval value:
1

;;; EC-EVAL input:
(fib 3)
(total-pushes = 21 maximum-depth = 6)

;;; EC-Eval value:
2

;;; EC-EVAL input:
(fib 4)
(total-pushes = 37 maximum-depth = 8)

;;; EC-Eval value:
3

;;; EC-EVAL input:
(fib 5)
(total-pushes = 61 maximum-depth = 10)

;;; EC-Eval value:
5

;;; EC-EVAL input:
(fib 6)
(total-pushes = 101 maximum-depth = 12)

;;; EC-Eval value:
8
\end{lstlisting}

Using the same reasoning as before we deduce the numbers of pushes to
compute $\mathrm{Fib}(n)$ is $8 \mathrm{Fib}(n+1)-3$ and the maximum
depth is $2 n$ for $n \ge 2$.

And finally for the special-purpose Fibonacci machine, we have

\begin{lstlisting}
> (define fib-machine
    (make-machine
     `((+ ,+) (- ,-) (< ,<) (read ,read))
     '(READ-LOOP
       (perform (op initialize-stack))
       (assign n (op read))
       (assign continue (label FIB-DONE))
       FIB-LOOP
       (test (op <) (reg n) (const 2))
       (branch (label IMMEDIATE-ANSWER))
       (save continue)
       (assign continue (label afterfib-n-1))
       (save n)
       (assign n (op -) (reg n) (const 1))
       (goto (label FIB-LOOP))
       AFTERFIB-N-1
       (restore n)
       (restore continue)
       (assign n (op -) (reg n) (const 2))
       (save continue)
       (assign continue (label AFTERFIB-N-2))
       (save val)
       (goto (label FIB-LOOP))
       AFTERFIB-N-2
       (assign n (reg val))
       (restore val)
       (restore continue)
       (assign val
               (op +) (reg val) (reg n))
       (goto (reg continue))
       IMMEDIATE-ANSWER
       (assign val (reg n))
       (goto (reg continue))
       fib-done
       (perform (op print-statistics))
       (goto (label READ-LOOP)))))

> (start fib-machine)
0
(total-pushes = 0 maximum-depth = 0)
1
(total-pushes = 0 maximum-depth = 0)
2
(total-pushes = 4 maximum-depth = 2)
3
(total-pushes = 8 maximum-depth = 4)
4
(total-pushes = 16 maximum-depth = 6)
5
(total-pushes = 28 maximum-depth = 8)
6
(total-pushes = 48 maximum-depth = 10)
\end{lstlisting}

If $S(n)$ denotes the number of pushes to compute $\mathrm{Fib(n)}$,
with the special-purpose register machine, we deduce easily that for
$n \ge 2$
\[ S(n) = S(n-1) + S(n-2) + 4.\]

Hence, using the same method as in \textbf{exercise 5.29}, we deduce
that
\[ S(n) = 4 \mathrm{Fib}(n+1) - 4\]

and the maximum depth is $2(n-1)$.

\medskip
To summarize the result, we have the following table for $n \ge 2$

\begin{quote}
  \begin{tabular}{|c|c|c|}
    \hline &Maximum depth& Number of pushes \\
    \hline Interpreted Fibonacci & $5n+3$ & $42\mathrm{Fib}(n+1)-30$ \\
    \hline Compiled Fibonacci & $2n$ & $8\mathrm{Fib}(n+1)-3$ \\
    \hline Fibonacci machine & $2(n-1)$ & $4\mathrm{Fib}(n+1)$ \\
    \hline
  \end{tabular}
\end{quote}

The ratios between the special-purpose machine and the interpreted
code tend to $2/5$ for the maximum depth and $2/21$ for the number of
pushes.  Whereas these numbers are respectively, $2/5$ and $4/21$
between the compiled code and the interpreted one.

Open-coding the $<$ primitive should allow the compiled code to be as
efficient as the special-purpose Fibonacci machine.

\end{document}
