\documentclass[a4paper,12pt]{article}
\usepackage{listings}
\lstset{language=Lisp}
\newcommand{\subpar}[1]{\medskip \noindent #1.}

\begin{document}

As usual, we add the following entry in \lstinline!analyze!

\begin{lstlisting}
  ...
  ((permanent-assignment? exp) (analyze-perm-assign exp))
  ...
\end{lstlisting}

And the we define \lstinline!analyze-perm-assign! as follows:

\begin{lstlisting}

(define (permanent-assignment? exp)
  (tagged-list? exp 'permanent-set!))

(define (analyze-perm-assign exp)
  (let ((var (assignment-variable exp))
        (vproc (analyze (assignment-value exp))))
    (lambda (env succeed fail)
      (vproc env
             (lambda (val fail2)
               (set-variable-value! var val env)
               (succeed 'ok fail2))
             fail))))
\end{lstlisting}

When we use \lstinline!permanent-set!!, the \lstinline!count! variable
contains the number of selected pairs $(x, y)$ up until now.
Whereas if we use \lstinline!set!!, each time we backtrack, the value
of \lstinline!count! will be reset to $0$.  Thus the third value in
the list being displayed is always equal to $1$.

\end{document}
