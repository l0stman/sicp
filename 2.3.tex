\documentclass[a4paper,12pt]{article}
\usepackage{listings}
\lstset{language=Lisp}

\begin{document}
We represent a rectangle with its width and length and construct
it with two perpendicular segments having the
same start point.
\begin{lstlisting}
(define (segment-length s)
  (sqrt (+ (square (- (x-point (start-segment s))
                      (x-point (end-segment s))))
           (square (- (y-point (start-segment s))
                      (y-point (end-segment s)))))))
(define (make-rectangle s1 s2)
  (let ((l (segment-length s1))
        (w (segment-length s2)))
    (cons (min w l) (max w l))))
(define (rectangle-length r) (cdr r))
(define (rectangle-width r) (car r))
\end{lstlisting}
We could then define the perimeter and area as follow
\begin{lstlisting}
(define (perimeter r)
  (+ (rectangle-length r) (rectangle-width r)))
(define (area r)
  (* (rectangle-length r) (rectangle-width r)))
\end{lstlisting}

\medskip \noindent
We could too represent the rectangle with two perpendicular segments
having the same starting point.
\begin{lstlisting}
(define (make-rectangle s1 s1) (cons s1 s2))
(define (rectangle-length r)
  (let ((l (segment-length (car r)))
        (w (segment-length (cdr r))))
    (max l w)))
(define (width-length r)
  (let ((l (segment-length (car r)))
        (w (segment-length (cdr r))))
    (min l w)))
\end{lstlisting}
This new representation doesn't change our definitions of
\lstinline!perimeter! and \lstinline!area!.
\end{document}
