\documentclass[a4paper,12pt]{article}
\usepackage{listings}
\newcommand{\ra}{\rightarrow}
\lstset{language=Lisp}

\begin{document}
\noindent
Let's replace \lstinline!timed-prime-test! by this less verbose
version:
\begin{lstlisting}
(define (timed-prime-test n)
  ((lambda (start-time)
     (when (fast-prime? n 10)
	   (- (current-inexact-milliseconds)
	      start-time)))
   (current-inexact-milliseconds)))
(define primes '((1009 1013 1019)
		 (10007 10009 10037)
		 (100003 100019 100043)
		 (1000003 1000033 1000037)))
\end{lstlisting}
We could then obain the list of the running time of each prime
\begin{lstlisting}
(define times
  (map (lambda (x)
	 (map timed-prime-test x))
       primes))
times
\end{lstlisting}
$\rightarrow$ 
((0.12890625
 0.116943359375
 0.1220703125) \\
 (0.156005859375
 0.156982421875
 0.15087890625) \\
 (0.64794921875
 0.64990234375 
 0.614990234375) \\
 (0.970947265625 
 1.01806640625
 1.113037109375)).

So taking the ratio of time of the first three primes with the last three
ones, we have:
\begin{lstlisting}
(map / (fourth times) (first times))
\end{lstlisting}
$\rightarrow$ (7.53219696969697 8.705636743215031 9.118).

Since the Fermat test has $\Theta(\log n)$ growth, this list of ration
should approach
\[ \frac{\log 10^6}{\log 10^3} = 2.\]
The discrepancy is due to the fact that the value $1000$ is too small
and the result is likely to be true only for large numbers.  For
example we have
\begin{lstlisting}
(map / (fourth times) (third times))
\end{lstlisting}
$\rightarrow$ (1.5688575899843507 1.4727731698521458
1.4708347886831994).

We see that the ratios is nearer to
\[ \frac{\log 10^6}{\log 10^5} = \frac{6}{5} = 1.2.\]
\end{document}
