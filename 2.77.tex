\documentclass[a4paper,12pt]{article}
\usepackage{listings}
\newcommand{\subpar}[1] {\medskip \noindent #1.}
\lstset{language=Lisp}

\begin{document}
It works because the first time \lstinline!apply-generic! is called,
it looks for the operator for complex numbers which is the same as the
generic operation on rectangular or polar numbers.

\begin{lstlisting}
  (magnitude z)
\end{lstlisting}

$\rightarrow$ \lstinline!(magnitude (complex rectangular 3 . 4))!

$\rightarrow$
\lstinline!(apply-generic magnitude (complex rectangular 3 . 4))!

$\rightarrow$
\lstinline!(apply (get magnitude (complex)) ((rectangular 3 . 4)))!

$\rightarrow$
\lstinline!(apply-generic magnitude (rectangular 3 . 4))!

$\rightarrow$
\lstinline!(apply (get magnitude (rectangular)) ((3 . 4)))!

$\rightarrow$ \lstinline!5!

\lstinline!apply-generic! is invoked twice and dispatched to the
procedure \lstinline!magnitude! for the current type of the number
each time.
\end{document}
