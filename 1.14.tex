\documentclass[a4paper,12pt]{article}
\begin{document}
Let's note $S(n, k)$ the space used by the process $cc$ on an amount
$n$ with $k$ kind of coins.  For large value of $n$ and $1 \le k\le
5$, we have
\[ S(n, k) = S(n - d_k, k) + S(n, k-1).\mbox{ (*)}\]
where $d_k$ is the value of the $k^{th}$ coin.  We have,
\[ S(n, 1) = S(n - 1, 1) + S(n, 0) = S(n-1, 1).\]
So we deduce easily that,
\[ S(n, 1) = \Theta(n).\]
Now suppose that: $S(n, k) = \Theta(n^k)$.
We then have from (*),
\begin{eqnarray*}
S(n, k+1) &=& S(n-d_{k+1}, k+1) + S(n, k) \\ 
	&=& S(n-d_{k+1}, k+1) + \Theta(n^k).
\end{eqnarray*}
We then deduce that,
\[ S(n, k+1) = \Theta(1) + \sum_{1 \le i \le n/d_{k+1}}
\Theta(i^k).\]
Thus,
\[ S(n, k+1) = \Theta(n^{k+1}).\]
Finally we then have,
\[ S(n, 5) = \Theta(n^5).\]
So the space used by the process is $\Theta(n^5)$.  And that number of
steps $T(n)$ being executed is proportionnal to the size of the tree being
generated, we have too $T(n) = \Theta(n^5)$.
\end{document}
