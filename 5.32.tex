\documentclass[a4paper,12pt]{article}
\usepackage{listings}
\lstset{language=Lisp}
\newcommand{\subpar}[1]{\medskip \noindent #1.}

\begin{document}

\subpar{a}  We just modify \lstinline!EV-APPLICATION! as follows
\begin{lstlisting}
EV-APPLICATION
(save continue)
(assign unev (op operands) (reg exp))
(assign exp (op operator) (reg exp))
(test (op variable?) (reg exp))
(branch (label EV-OPERATOR-VARIABLE))
(save env)
(save unev)
(assign continue (label EV-APPL-DID-OPERATOR))
(goto (label EVAL-DISPATCH))
EV-OPERATOR-VARIABLE
(assign proc (op lookup-variable-value)
             (reg exp)
             (reg env))
(goto (label EV-APPL-PROC-ASSIGNED))
EV-APPL-DID-OPERATOR
(restore unev)                          ; the operands
(restore env)
(assign proc (reg val))                 ; the operator
EV-APPL-PROC-ASSIGNED
(assign argl (op empty-arglist))
(test (op no-operands?) (reg unev))
(branch (label APPLY-DISPATCH))
(save proc)
EV-APPL-OPERAND-LOOP
\end{lstlisting}

\subpar{b} First of all, the advantage of an interpreter compared to a
compiler is the low overhead when starting a program.  If an
interpreter does too much analyzes before evaluating an expression, we
lose this advantage.

Secondly even if what Alyssa says is true in theory, it would be
impractical since the interpreter would become too bloated.  Suppose
that we separate the syntactic analysis from execution like in 4.1.7
to have a more efficient interpreter.  And suppose we want to
integrate into the interpreter the optimizations in \textbf{exercise
  5.31}.  It means that when analyzing an application expression, we
need to dispatch the evaluation of the operator and each operands to
the optimal version of subroutine that minimizes saves and restores.

For an application with four arguments, we thus need to implement $2^4
= 16$ subroutines according to the fact that we need to preserve the
register \textbf{env} or not (we don't need to save it for the last
operand).  And we also need to take into account the registers
\textbf{proc} and \textbf{argl}.  It's easy to see that the
implementation is really cumbersome, so if we need to do these
implementations, a compiler is the way to go.

\end{document}
