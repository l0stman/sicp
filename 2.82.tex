\documentclass[a4paper,12pt]{article}
\usepackage{listings}
\newcommand{\subpar}[1] {\medskip \noindent #1.}
\lstset{language=Lisp}

\begin{document}
\begin{lstlisting}
(define (coerce-lst args type-tags)
  (define (totype type args res)
    (if (null? args)
        (reverse res)
        (let ((old-type (type-tag (car args))))
          (if (eq? type old-type)
              (totype type (cdr args) (cons (car args) res)) 
              (let ((obj->type (get-coercion old-type type)))
                (if obj->type
                    (totype type
                            (cdr args)
                            (cons (obj->type (car args)) res))
                    #f))))))
  (define (iter type-tags)
    (if (null? type-tags)
        #f
        (let ((lst (totype (car type-tags) args '())))
          (if lst
              lst
              (iter (cdr type-tags))))))
  (iter type-tags))

(define (apply-generic op . args)
  (let ((type-tags (map type-tag args)))
    (let ((proc (get op type-args)))
      (if proc
          (apply proc (map contents args))
          (let ((lst (coerce-lst args type-tags)))
            (if lst
                (apply-generic op lst)
                (error "No method for these types"
                       (list op type-tags))))))))
\end{lstlisting}
This strategy is not sufficiently general.  Suppose we have
implemented a binary operation only for complex numbers.  If the
arguments are a scheme-number and a rational,
\lstinline!apply-generic! will fail although both these numbers could
be coerced into a complex one.
\end{document}
