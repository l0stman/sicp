\documentclass[a4paper,12pt]{article}
\usepackage{listings}
\lstset{language=Lisp}

\begin{document}
With mzscheme, we define \textbf{runtime} as
\textbf{current-inexact-milliseconds}.  Here's
the code for \textbf{search-for-primes}.
\begin{lstlisting}
(define (search-for-primes init range)
  (define (loop n range)
    (when (<= (- n init) range)
	  (timed-prime-test n)
	  (iter (+ n 2) range)))
  (if (even? init)
      (loop (+ init 1) (- range 1))
      (loop init range)))
\end{lstlisting}
Here are the first three lines in the results we obtain:
\begin{eqnarray*}
\lstinline!(search-for-primes 1000 100)!
&\rightarrow&
1009 \mbox{ *** }0.012939453125 \\ &&
1013 \mbox{ *** }0.012939453125 \\ &&
1019 \mbox{ *** }0.012939453125 \\
\lstinline!(search-for-primes 10000 100)!
&\rightarrow&
10007 \mbox{ *** } 0.0380859375 \\ &&
10009 \mbox{ *** } 0.036865234375 \\ &&
10037 \mbox{ *** } 0.036865234375 \\
\lstinline!(search-for-primes 100000 100)!
&\rightarrow&
100003 \mbox{ *** } 0.125 \\ &&
100019 \mbox{ *** } 0.11181640625 \\ &&
100043 \mbox{ *** } 0.113037109375 \\
\lstinline!(search-for-primes 1000000 100)!
&\rightarrow&
1000003\mbox{ *** } 0.39697265625 \\ &&
1000033\mbox{ *** } 0.342041015025 \\ &&
1000037\mbox{ *** } 0.35498046875
\end{eqnarray*}
Taking the average of the time elapsed for the three prime numbers
found and computing the successive ratio, we
obtion the sequel: $2.88$, $3.13$, $3.13$.  So given that $\sqrt{10}
\simeq 3.16$, our result supports the $\sqrt{n}$ growth prediction.
So the programs on the machine run in time proportional to the number
of steps required for the computation.
\end{document}
