\documentclass[a4paper,12pt]{article}
\usepackage{listings}
\newcommand{\subpar}[1] {\medskip \noindent #1.}
\lstset{language=Lisp}

\begin{document}
We add the following definition in the complex numbers package:
\begin{lstlisting}
(define (project c)
  (make-scheme-number (real-part c)))
(put 'project '(complex) project)
\end{lstlisting}
And add the following pieces of code in the rational number package:
\begin{lstlisting}
(define (project r)
  (make-scheme-number (numer r)))
(put 'project '(rational project))
\end{lstlisting}
And finally in the scheme-number package:
\begin{lstlisting}
(define (project n)
  (make-scheme-number (round n)))
(put 'project 'scheme-number project)
\end{lstlisting}
We then define the generic operation \lstinline!project! and the
procedure \lstinline!drop!.
\begin{lstlisting}
(define (project n)
  (apply-generic 'project n))

(define (drop n)
  (let ((p (project n)))
    (if (equ? (raise p) n)
        (drop p)
        n)))
\end{lstlisting}
And we redefine \lstinline!apply-generic-raise! to simplify its
results.
\begin{lstlisting}
  (define (apply-generic-raise op args init)
  (let* ((type-tags (map type-tag args))
         (proc (get op type-tags)))
    (if proc
        (drop (apply proc (map contents args)))
        (let ((objs (raise-lst args init)))
          (if objs
              (apply-generic-raise op objs init)
              (error
               "No method for these types"
               (list op (map type-tag init))))))))
\end{lstlisting}
\end{document}
