\documentclass[a4paper,12pt]{article}
\newcommand{\subpar}[1]{\medskip \noindent #1.}
\newcommand{\la}{\leftarrow}
\usepackage{listings}
\usepackage{pgf}
\usepgfmodule{plot,shapes,matrix}
\usepackage{tikz}
\usetikzlibrary{matrix,positioning,shapes,arrows,calc}

\makeatletter
\pgfdeclareshape{forbidden parking}
{
  \inheritsavedanchors[from=circle] % this is nearly a circle
  \inheritanchorborder[from=circle]
  \inheritanchor[from=circle]{north}
  \inheritanchor[from=circle]{north west}
  \inheritanchor[from=circle]{north east}
  \inheritanchor[from=circle]{center}
  \inheritanchor[from=circle]{west}
  \inheritanchor[from=circle]{east}
  \inheritanchor[from=circle]{mid}
  \inheritanchor[from=circle]{mid west}
  \inheritanchor[from=circle]{mid east}
  \inheritanchor[from=circle]{base}
  \inheritanchor[from=circle]{base west}
  \inheritanchor[from=circle]{base east}
  \inheritanchor[from=circle]{south}
  \inheritanchor[from=circle]{south west}
  \inheritanchor[from=circle]{south east}
  \inheritbackgroundpath[from=circle]
  \foregroundpath{
    \centerpoint%
    \pgf@xc=\pgf@x%
    \pgf@yc=\pgf@y%
    \pgfutil@tempdima=\radius%
    \pgfmathsetlength{\pgf@xb}{\pgfkeysvalueof{/pgf/outer xsep}}%
    \pgfmathsetlength{\pgf@yb}{\pgfkeysvalueof{/pgf/outer ysep}}%
    \ifdim\pgf@xb<\pgf@yb%
      \advance\pgfutil@tempdima by-\pgf@yb%
    \else%
      \advance\pgfutil@tempdima by-\pgf@xb%
    \fi%
    \pgfpathmoveto{\pgfpointadd{\pgfqpoint{\pgf@xc}{\pgf@yc}}{\pgfqpoint{-0.707107\pgfutil@tempdima}{-0.707107\pgfutil@tempdima}}}
    \pgfpathlineto{\pgfpointadd{\pgfqpoint{\pgf@xc}{\pgf@yc}}{\pgfqpoint{0.707107\pgfutil@tempdima}{0.707107\pgfutil@tempdima}}}
    \pgfpathmoveto{\pgfpointadd{\pgfqpoint{\pgf@xc}{\pgf@yc}}{\pgfpoint{-0.707107\pgfutil@tempdima}{0.707107\pgfutil@tempdima}}}
    \pgfpathlineto{\pgfpointadd{\pgfqpoint{\pgf@xc}{\pgf@yc}}{\pgfqpoint{0.707107\pgfutil@tempdima}{-0.707107\pgfutil@tempdima}}}
  }
}
\makeatother

\newenvironment{datapath}
               {
                 \begin{tikzpicture}
                   [
                     >=stealth',
                     constant/.style={
                       isosceles triangle,
                       isosceles triangle apex angle=60,
                       draw,
                     },
                     register/.style={
                       rectangle,
                       minimum width=15mm,
                       draw,
                       text height=1.1ex,
                       text depth=.20ex,
                     },
                     button/.style={
                       forbidden parking,
                       inner sep=3pt,
                       draw
                     },
                     operation/.style={
                       trapezium,
                       draw,
                       text height=1.1ex,
                       text depth=.20ex,
                     },
                     test/.style={
                       circle,
                       draw
                     },
                     point/.style={
                       circle,
                       fill,
                       draw,
                       inner sep=1.5pt
                     },
                     every label/.style={font=\scriptsize}
                   ]
               }
               {\end{tikzpicture}}

\newenvironment{controller}
               {
                 \begin{tikzpicture}
                   [
                     font=\scriptsize,
                     button/.style={rectangle,draw},
                     test/.style={diamond,draw},
                   ]
               }
               {\end{tikzpicture}}

\begin{document}

Here's the data path of a register machine computing factorials.

\begin{datapath}
  \matrix[row sep=5mm,column sep=5mm] {
    & & \node[constant,shape border rotate=-90] (one) {$1$}; \\
    \node[register] (n) {n}; & & \node[button]
         [label=right:count$\la$1] (counter init) {}; &
    & \node[button] (product init) [label=right:prod$\la$1] {}; \\
    \node[test] (compare) {$>$}; & & \node[register] (counter)
         {counter}; & & \node[register] (product) {product}; \\
    & \node[operation, shape border rotate=180] (add) {$+$}; &
    \node[button] (update counter) [label=right:count$\la+$1] {}; &
    \node[operation,shape border
      rotate= 180] (mul) {$*$}; &
    \node[button] (update product) [label=right:prod$\la*$count] {}; \\
  };

  \node [point] (one inter) at ($(one.south) + (0,-.25)$) {};
  \node [point] at (add.north east |- counter.west) {};

  \path (one) edge (one inter)
        (one inter) edge (counter init)
        (counter init) edge[->] (counter)
        (update counter) edge[->] (counter)
        (product init) edge[->] (product)
        (update product) edge[->] (product)
        (n) edge[->] (compare)
        (counter) edge[->] (compare);

  \draw (one inter) -| (product init.north);
  \draw (one inter) -| (add.north west);
  \draw (add.north east |- counter.west) -- (add.north east);
  \draw (counter.east) -| (mul.north west);
  \draw (product.west) -| (mul.north east);
  \draw (add.south) -- ++(0,-.5) -| (update counter.south);
  \draw (mul.south) -- ++(0,-.5) -| (update product.south);
\end{datapath}

And here's the corresponding controller diagrams

\begin{controller}
  \node (start) {start};
  \node [button] (prod init) [below=of start] {prod$\la1$};
  \node [button] (count init) [below=of prod init] {count$\la1$};
  \node [test] (compare) [below=of count init,label=above
    right:yes,label=below right:no] {$>$};
  \node [button] (prod update) [below=of compare] {prod$\la*$count};
  \node [button] (count update) [below=of prod update] {count$\la+1$};
  \node (done) [right=of compare] {done};

  \path (start) edge[->] (prod init)
        (prod init) edge[->] (count init)
        (count init) edge[->] (compare)
        (compare) edge[->] (prod update)
        (prod update) edge[->] (count update)
        (compare) edge[->] (done);

\draw (count update.west) -- ++(-.5,0) [->] |- (compare.west) ;
\end{controller}

\end{document}
