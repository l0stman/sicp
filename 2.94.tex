\documentclass[a4paper,12pt]{article}
\usepackage{listings}
\newcommand{\subpar}[1] {\medskip \noindent #1.}
\lstset{language=Lisp}

\begin{document}
\begin{lstlisting}
(define (remainder-terms a b)
  (cadr (div-terms a b)))
(define (gcd-terms a b)
  (if (empty-termlist? b)
      a
      (gcd-terms b (remainder-terms a b))))
(define (gcd-poly p1 p2)
  (if (same-variable? (variable p1) (variable p2))
      (make-poly (variable p1)
                 (gcd-terms (term-list p1)
                            (term-list p2)))
      (error "Polys not in same var -- MUL-POLY"
             (list p1 p2))))
\end{lstlisting}
We could then define a generic operation
\lstinline!greatest-common-divisor! that reduces to
\lstinline!gcd-poly! for polynomials and ordinary \lstinline!gcd! for
ordinary numbers:
\begin{lstlisting}
(define (greatest-common-divisor a b)
  (applygeneric 'greatest-common-divisor a b))
\end{lstlisting}
So we add the following definition in the \emph{scheme number}
package:
\begin{lstlisting}
(put 'greatest-common-divisor
     '(scheme-number scheme-number)
     gcd)
\end{lstlisting}
And in the \emph{polynomial package}:
\begin{lstlisting}
(put 'greatest-common-divisor
     '(polynomial polynomial)
     gcd-poly)
\end{lstlisting}
\end{document}
