\documentclass[a4paper,12pt]{article}
\usepackage{listings}
\lstset{language=Lisp}
\newcommand{\subpar}[1]{\medskip \noindent #1.}

\begin{document}

We add to the new register machine an operator \lstinline!compile!
that returns the assembled instructions for a given expression.
Otherwise, we use the same operations as the explicit-control evaluator:

\begin{lstlisting}
(define (compile exp)
  (assemble
   (statements
    (compile* exp
              'val
              'return
              the-empty-comp-time-env))
   eccompile))

(define eccompile
  (make-machine
   (cons (list 'compile compile) eceval-operations)
   '(READ-COMPILE-EXECUTE-PRINT-LOOP
     (perform (op initialize-stack))
     (perform (op prompt-for-input)
              (const ";;; EC-COMPILE input:"))
     (assign exp (op read))
     (assign env (op get-global-environment))
     (assign val (op compile) (reg exp))
     (assign continue (label PRINT-RESULT))
     (goto (reg val))
     PRINT-RESULT
     (perform (op print-statistics))
     (perform (op announce-output)
              (const ";;; EC-COMPILE value:"))
     (perform (op user-print) (reg val))
     (goto (label READ-COMPILE-EXECUTE-PRINT-LOOP)))))
\end{lstlisting}

Here's a simulation

\begin{lstlisting}
> (start eccompile)


;;; EC-COMPILE input:
(define (fact n)
  (if (= n 1)
      1
      (* (fact (- n 1)) n)))
(total-pushes = 0 maximum-depth = 0)

;;; EC-COMPILE value:
ok

;;; EC-COMPILE input:
(fact 3)
(total-pushes = 10 maximum-depth = 6)

;;; EC-COMPILE value:
6

;;; EC-COMPILE input:
(fact 4)
(total-pushes = 14 maximum-depth = 8)

;;; EC-COMPILE value:
24
\end{lstlisting}

\end{document}
