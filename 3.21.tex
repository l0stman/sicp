\documentclass[a4paper,12pt]{article}
\usepackage{listings}
\lstset{language=Lisp}

\begin{document}
What Ben sees on the repl is the internal representation of the queue.
So it only makes sense if he uses the predefined accessors.  In this
case, if there's only one element left in the queue, both the front
and rear pointers point to the same element.  But deleting this
element will only set the front pointer to an empty list, the rear
pointer still points to the element.  But this is fine since
\lstinline!empty-queue?! only tests if the front pointer is empty.
\begin{lstlisting}
(define (print-queue q)
  (if (empty-queue? q)
      '()
      (front-ptr q)))
\end{lstlisting}
\end{document}
