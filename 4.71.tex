\documentclass[a4paper,12pt]{article}
\usepackage{listings}
\lstset{language=Lisp}
\newcommand{\subpar}[1]{\medskip \noindent #1.}

\begin{document}

Given that the order of evaluation of the arguments to a Scheme
procedure are undefined, we should enforce to have some desired
behaviors.

For example, suppose that a particular Scheme implementation evaluates
the arguments from right to left.  And \lstinline!simple-query! is
defined as Louis Reasoner's implementation.  Thus the rules are
evaluated before the assertions.  Moreover, we define the following
assertion and rule

\begin{lstlisting}
(assert! (married Minnie Mickey))
(assert! (rule (married ?x ?y)
               (married ?y ?x)))
\end{lstlisting}

With the former definition, the query
\lstinline!(married Mickey ?who)!  would print the result
\begin{lstlisting}
  (married Mickey Minnie)
\end{lstlisting}
and then goes into an infinite loop.  Whereas with the new definition
of \lstinline!simple-query!, no result at all would be printed and the
system will loop indefinitely.  Thus evaluating the assertions before
the rules would at least ensure us that if the query verifies some
assertions in the database, they would be printed even if the system
would go into an infinite loop.

\medskip
The same reasoning could be applied to \lstinline!disjoin!.  Ensuring
an order of evaluation in the queries would at least guarantees that
if the first queries produce some result, they would be printed even
if the latter queries cause the system to not terminate.

\end{document}
