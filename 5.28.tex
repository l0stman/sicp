\documentclass[a4paper,12pt]{article}
\usepackage{listings}
\lstset{language=Lisp}
\newcommand{\subpar}[1]{\medskip \noindent #1.}

\begin{document}

By modifying \lstinline!eval-sequence! as described in section 5.4.2
so that the evaluator is no longer tail-recursive, we obtain the
following result

\begin{lstlisting}
;;; EC-EVAL input:
(define (fact-iter n)
  (define (iter product counter)
    (if (> counter n)
        product
        (iter (* counter product)
              (+ counter 1))))
  (iter 1 1))
(total-pushes = 3 maximum-depth = 3)

;;; EC-Eval value:
ok

;;; EC-EVAL input:
(fact-iter 1)
(total-pushes = 70 maximum-depth = 17)

;;; EC-Eval value:
1

;;; EC-EVAL input:
(fact-iter 2)
(total-pushes = 107 maximum-depth = 20)

;;; EC-Eval value:
2

;;; EC-EVAL input:
(fact-iter 3)
(total-pushes = 144 maximum-depth = 23)

;;; EC-Eval value:
6

;;; EC-EVAL input:
(fact-iter 4)
(total-pushes = 181 maximum-depth = 26)

;;; EC-Eval value:
24

;;; EC-EVAL input:
(define (fact-recur n)
  (if (= n 1)
      1
      (* (fact-recur (- n 1)) n)))
(total-pushes = 3 maximum-depth = 3)

;;; EC-Eval value:
ok

;;; EC-EVAL input:
(fact-recur 1)
(total-pushes = 18 maximum-depth = 11)

;;; EC-Eval value:
1

;;; EC-EVAL input:
(fact-recur 2)
(total-pushes = 52 maximum-depth = 19)

;;; EC-Eval value:
2

;;; EC-EVAL input:
(fact-recur 3)
(total-pushes = 86 maximum-depth = 27)

;;; EC-Eval value:
6

;;; EC-EVAL input:
(fact-recur 4)
(total-pushes = 120 maximum-depth = 35)

;;; EC-Eval value:
24
\end{lstlisting}

From these numbers we induce the following result
\begin{quote}
  \begin{tabular}{|c|c|c|}
    \hline &Maximum depth& Number of pushes \\
    \hline Recursive factorial& $8 n + 3$ & $34 n - 16$ \\
    \hline Iterative factorial& $3 n + 14$ & $37 n + 33$ \\
    \hline
  \end{tabular}
\end{quote}

\end{document}
