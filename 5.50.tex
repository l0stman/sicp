\documentclass[a4paper,12pt]{article}
\usepackage{listings}
\lstset{language=Lisp}
\newcommand{\subpar}[1]{\medskip \noindent #1.}

\begin{document}

We try to compile the meta-circular evaluator defined in the file
eval.scm by wrapping it inside a \lstinline!begin! statement.  First
of all, booleans should be recognized  as self-evaluating expressions
so the compiler won't choke:
\begin{lstlisting}
(define (self-evaluating? exp)
  (cond ((number? exp) #t)
        ((string? exp) #t)
        ((boolean? exp) #t)
        (else #f)))
\end{lstlisting}

High order functions also pose problems since we have our own
procedure implementation.  Hence, we can't use function such as
\lstinline!map! as primitive procedure.  Here we redefine them:

\begin{lstlisting}
(define (map fn lst)
     (if (null? lst)
         '()
         (cons (fn (car lst))
               (map fn (cdr lst)))))

(define (for-each fn lst)
  (if (null? lst)
      'done
      (begin
        (fn (car lst))
        (for-each fn (cdr lst)))))
\end{lstlisting}

Here's a simulation of the compiled version of the meta-circular
evaluator wrapped inside a begin:

\begin{lstlisting}
> (compile-and-go
   '(begin
      <meta-circular evaluator body>))
(total-pushes = 2662 maximum-depth = 156)

;;; EC-EVAL value:
ok

;;; EC-EVAL input:
1
(total-pushes = 0 maximum-depth = 0)

;;; EC-EVAL value:
1

;;; EC-EVAL input:
#t
(total-pushes = 0 maximum-depth = 0)

;;; EC-EVAL value:
#t

;;; EC-EVAL input:
(define (fact n)
  (if (= n 1)
      1
      (* (fact (- n 1)) n)))
(total-pushes = 3 maximum-depth = 3)

;;; EC-EVAL value:
ok

;;; EC-EVAL input:
(fact 2)
(total-pushes = 36 maximum-depth = 13)

;;; EC-EVAL value:
2

;;; EC-EVAL input:
(fact 4)
(total-pushes = 84 maximum-depth = 23)

;;; EC-EVAL value:
24
\end{lstlisting}

\end{document}
