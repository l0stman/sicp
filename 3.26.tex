\documentclass[a4paper,12pt]{article}
\usepackage{listings}
\lstset{language=Lisp}

\begin{document}
Let's introduce the data structure representing a node:
\begin{lstlisting}
(define empty-node '())
(define (empty-node? n) (eq? n empty-node))
(define (make-node key val)
  (list 'node key val empty-node empty-node))
(define (node? n)
  (and (pair? n) (eq? (car n) 'node)))
(define node-key cadr)
(define node-val caddr)
(define node-left cadddr)
(define node-right (compose cadddr cdr))
(define (set-val! node val) (set-car! (cddr node) val))
(define (set-left-node! node n) (set-car! (cdddr node) n))
(define (set-right-node! node n) (set-car! (cddddr node) n))
\end{lstlisting}
The constructor \lstinline!make-table! takes as argument a comparison
procedure for keys returning respectively 0, a negative number and a
positive number if the first key is equal, less, or greater than the
second one.
\begin{lstlisting}
(define (make-table cmp)
  (let ((local-table empty-node))
    (define (find key table)
      (if (empty-node? table)
          #f
          (let ((n (cmp key (node-key table))))
            (cond ((= n 0) table)
                  ((< n 0) (find key (node-left table)))
                  (else (find key (node-right table)))))))
    (define (lookup keys table)
      (cond ((empty-node? table) #f)
            ((null? keys) table)
            (else
             (let ((subtable (find (car keys) table)))
               (cond ((not subtable) #f)
                     ((null? (cdr keys)) (node-val subtable))
                     (else
                      (lookup (cdr keys) (node-val subtable))))))))
    (define (insert1! key val table)
      (let ((new-node (make-node key val)))
       (if (empty-node? local-table)
           (set! local-table new-node)
           (let ((n (cmp key (node-key table))))
             (cond ((= n 0) (set-val! table val))
                   ((< n 0)
                    (if (empty-node? (node-left table))
                        (set-left-node! table new-node)
                        (insert1! key val (node-left table))))
                   (else
                    (if (empty-node? (node-right table))
                        (set-right-node! table new-node)
                        (insert1! key val (node-right table)))))))))
    (define (insert! keys val table)
      (if (null? (cdr keys))
          (insert1! (car keys) val table)
          (let ((subtable (find (car keys) table)))
            (if (and subtable (node? (node-val subtable)))
                (insert! (cdr keys) val (node-val subtable)) 
                (insert1! (car keys)
                          (foldl make-node val (nreverse (cdr keys)))
                          table))))
      'ok)
    (define (dispatch m)
      (cond ((eq? m 'lookup-proc)
             (lambda (keys) (lookup keys local-table)))
            ((eq? m 'insert-proc!)
             (lambda (keys val) (insert! keys val local-table)))
            (else
             (error "Unknown operation -- TABLE" m))))
      dispatch))
\end{lstlisting}
For example, if the keys are numbers, we can built a table as follows:
\begin{lstlisting}
(define *table* (make-table (lambda (x y) (- x y))))
(define put (*table* 'insert-proc!))
(define get (*table* 'lookup-proc))
\end{lstlisting}
\end{document}
