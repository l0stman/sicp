\documentclass[a4paper,12pt]{article}
\usepackage{listings}
\lstset{language=Lisp}

\begin{document}
\begin{lstlisting}
(define (adjoin-set x set)
  (define (insert-iter head tail)
    (cond ((null? tail)
           (reverse (cons x head)))
          ((= x (car tail)) set)
          ((< x (car tail))
           (foldl cons (cons x tail) head))
          (else
           (insert-iter (cons (car tail) head)
                        (cdr tail)))))
  (insert-iter '() set))

(define (adjoin-set x set)
  (cond ((null? set) (list x))
        ((= x (car set)) set)
        ((< x (car set)) (cons x set))
        (else (cons (car set)
                    (adjoin-set x (cdr set))))))
\end{lstlisting}
For the element isn't in the set, it will be added at the end of the
list, so it'll take roughly 2n steps if the set is ordered or not.
But on the average, we should expect to have to examine about half of
the items in the set.   And adding the element will cost on average
$n/2$, thus on average the running time is $n$.
\end{document}
