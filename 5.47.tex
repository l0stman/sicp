\documentclass[a4paper,12pt]{article}
\usepackage{listings}
\lstset{language=Lisp}
\newcommand{\subpar}[1]{\medskip \noindent #1.}

\begin{document}

First we modify the procedure call produced by the compiler to
generate another branch for an interpreted code.

\begin{lstlisting}
(define (compile-procedure-call target linkage)
  (let ((primitive-branch
         (make-label 'primitive-branch))
        (compiled-branch
         (make-label 'compiled-branch))
        (interpreted-branch
         (make-label 'interpreted-branch))
        (after-call (make-label 'after-call)))
    (let ((nonprim-linkage
           (if (eq? linkage 'next)
               after-call
               linkage)))
      (append-instruction-sequences
       (make-instruction-sequence
        '(proc)
        '()
        `((test (op primitive-procedure?) (reg proc))
          (branch (label ,primitive-branch))
          (test (op compiled-procedure?) (reg proc))
          (branch (label ,compiled-branch))))
       (parallel-instruction-sequences
        (append-instruction-sequences
         interpreted-branch
         (interpreted-proc-appl target
                                nonprim-linkage))
        (parallel-instruction-sequences
         (append-instruction-sequences
          compiled-branch
          (compile-proc-appl target nonprim-linkage))
         (append-instruction-sequences
          primitive-branch
          (end-with-linkage
           linkage
           (make-instruction-sequence
            '(proc argl)
            (list target)
            `((assign ,target
                      (op apply-primitive-procedure)
                      (reg proc)
                      (reg argl))))))))
       after-call))))
\end{lstlisting}

Then we generate special code for an interpreted procedure application
to save the \textbf{continue} register on the stack and jump to
\lstinline!COMPOUND-APPLY! in the evaluator whose address is saved in
the register \textbf{compapp}.

\begin{lstlisting}
(define (interpreted-proc-appl target linkage)
  (cond ((and (eq? target 'val)
              (not (eq? linkage 'return)))
         (make-instruction-sequence
          '(proc)
          all-regs
          `((assign continue (label ,linkage))
            (save continue)
            (goto (reg compapp)))))
        ((and (not (eq? target 'val))
              (not (eq? linkage 'return)))
         (let ((proc-return (make-label 'proc-return)))
           (make-instruction-sequence
            '(proc)
            all-regs
            `((assign continue (label ,proc-return))
              (save continue)
              (goto (reg compapp))
              ,proc-return
              (assign ,target (reg val))
              (goto (label ,linkage))))))
        ((and (eq? target 'val) (eq? linkage 'return))
         (make-instruction-sequence
          '(proc)
          all-regs
          '((save continue)
            (goto (reg compapp)))))
        ((and (not (eq? target 'val))
              (eq? linkage 'return))
         (error
          "return linkage, target not val -- COMPILE"
          target))))
\end{lstlisting}

Here are some codes testing this new feature:
\begin{lstlisting}
> (compile-and-go
   '(define (f x)
      (+ 1 (g x))))
(total-pushes = 0 maximum-depth = 0)

;;; EC-EVAL value:
ok

;;; EC-EVAL input:
(define (g x)
  (* x x))
(total-pushes = 3 maximum-depth = 3)

;;; EC-EVAL value:
ok

;;; EC-EVAL input:
(f 3)
(total-pushes = 12 maximum-depth = 7)

;;; EC-EVAL value:
10

;;; EC-EVAL input:
(f 4)
(total-pushes = 12 maximum-depth = 7)

;;; EC-EVAL value:
17
\end{lstlisting}

\end{document}
