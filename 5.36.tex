\documentclass[a4paper,12pt]{article}
\usepackage{listings}
\lstset{language=Lisp}
\newcommand{\subpar}[1]{\medskip \noindent #1.}

\begin{document}

The compiler produce for operands of a combination an order of
evaluation from right-to-left.  The order is determined by the
procedure \lstinline!construct-arglist!.  Let's modify
\lstinline!construct-arglist! such that the order of evaluation
becomes left-to-right using the operator \lstinline!adjoin-arg!
defined in 5.4.1.

\begin{lstlisting}
(define (construct-arglist operand-codes)
  (let ((init-argl
         (make-instruction-sequence
          '()
          '(argl)
          '((assign argl (const ()))))))
    (if (null? operand-codes)
        init-argl
        (append-instruction-sequences
         init-argl
         (code-to-get-args operand-codes)))))
(define (code-to-get-args operand-codes)
  (let ((code-for-next-arg
         (preserving
          '(argl)
          (car operand-codes)
          (make-instruction-sequence
           '(val argl)
           '(argl)
           '((assign argl
                     (op adjoin-arg)
                     (reg val)
                     (reg argl)))))))
    (if (null? (cdr operand-codes))
        code-for-next-arg
        (preserving
         '(env)
         code-for-next-arg
         (code-to-get-args (cdr operand-codes))))))
\end{lstlisting}

The newly generated code is less efficient than the previous one since
constructing  the arguments list become $\Theta(n^2)$ instead of
$\Theta(n)$.

\end{document}
