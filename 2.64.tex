\documentclass[a4paper,12pt]{article}
\usepackage{listings}
\newcommand{\subpar}[1] {\medskip \noindent #1.} \lstset{language=Lisp}

\begin{document}
\subpar{a}  Basically, \lstinline!partial-tree! built a balanced tree
with the first $\lfloor (n-1)/2 \rfloor$ elements of \lstinline!elts!,
this will be the left branch of the newly built balanced tree.  Choose
the $\lfloor (n-1)/2\rfloor + 1$ element as the entry, and built a
balanced tree with the next $\lceil (n-1)/2\rceil$ elements and make it the
right branch.  This build a balanced tree with the first $n$ elements
of \lstinline!elts!.  It then returns a pair containing this tree as
first element and the remainder of \lstinline!els! as second element.

\subpar{b}  Note $T$ the running time of \lstinline!partial-tree!.  We
have the recurrence,
\[ T(n) = T\left(\left\lfloor\frac{n-1}{2}\right\rfloor\right) +
T\left(\left\lceil\frac{n-1}{2}\right\rceil\right) + \Theta(1).\]
If we're being sloppy, this recurrence could be rewritten as
\[ T(n) = 2T\left(\frac{n}{2}\right) + \Theta(1).\]
Thus, we have $T(n) = \Theta(n)$.  Let's show that this is really the
case without being sloppy.

Let's show that $T(n) \le c\,n$ for an appropriate choice of the
constant $c>0$.  Suppose we have the inequality for
$\left\lfloor\frac{n-1}{2}\right\rfloor$ and
$\left\lceil\frac{n-1}{2}\right\rceil$, we have
\begin{eqnarray*}
  T(n) &=& T\left(\left\lfloor\frac{n-1}{2}\right\rfloor\right) +
  T\left(\left\lceil\frac{n-1}{2}\right\rceil\right) + \Theta(1) \\
  &\le& c\left\lfloor\frac{n-1}{2}\right\rfloor +
  c\left\lceil\frac{n-1}{2}\right\rceil + \Theta(1) \\
  &=& c(n-1) + \Theta(1) \\
  &=& c\,n - (c - \Theta(1)) \\
  &\le& c\,n
\end{eqnarray*}
if we have $c \ge \Theta(1)$.  Plus $c$ should be large enough
such that $T(1) \le c$.  We then deduce that $T(n) = O(n)$.  Using
the same reasoning, we could show that $T(n) = \Omega(n)$ such that
$T(n) = \Theta(n)$.
\end{document}
