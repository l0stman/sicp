\documentclass[a4paper,12pt]{article}
\usepackage{listings}
\lstset{language=Lisp}
\newcommand{\subpar}[1]{\medskip \noindent #1.}

\begin{document}

If the first stream is infinite, no element of the second stream would
never appear if we just use \lstinline!stream-append!.  Whereas
interleaving the two streams assures that given enough time, any
element in the two streams would eventually appears in the resulting
stream.

Remember that with our implementation, the rules are implemented using
a LIFO order.  Consider the following example:

\begin{lstlisting}
(assert! (rule (all ?color (?color . ?rest))
               (all ?color ?rest)))
(assert! (rule (all ?color ())))
(assert! (rule (unicolor ?x)
               (or (all white ?x)
                   (all black ?x))))
\end{lstlisting}

If \lstinline!disjoin! didn't interleave the streams, no matter how
long we're waiting, it seems only the lists whose elements are all
equal to \lstinline!white! that are considered \lstinline!unicolor!.
\end{document}
