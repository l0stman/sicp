\documentclass[a4paper,12pt]{article}
\usepackage{listings}
\lstset{language=Lisp}

\begin{document}
\noindent
a. Here's an iterative procedure defining \lstinline!cont-frac!
\begin{lstlisting}
(define (cont-frac n d k)
  (define (iter res k)
    (if (= k 0)
	res
	(iter (/ (n k)
		 (+ (d k) res))
	      (- k 1))))
  (iter 0 k))
\end{lstlisting}
Let's check it for the inverse of the golden ratio
\begin{lstlisting}
(define inv-phi
  (/ 1
     (/ (+ 1 (sqrt 5))
	2)))
(define (f k)
  (cont-frac (lambda (i) 1.0)
	     (lambda (i) 1.0)
	     k))
inv-phi
\end{lstlisting}
$\rightarrow$ 0.6180339887498948.
\begin{lstlisting}
(f 4)
\end{lstlisting}
$\rightarrow$ 0.6000000000000001
\begin{lstlisting}
(f 12)
\end{lstlisting}
$\rightarrow$ 0.6180257510729613

\medskip We can see that the successive value of the procedure
\lstinline!f! approaches the inverse of the golden ratio.

To find the value of $k$ in order to get an approximation that is
accurate to $4$ decimal places we define the following procedure
\begin{lstlisting}
(define (find k acc)
  (let ((x (f k)))
    (if (< (abs (- x inv-phi))
	   acc)
	k
	(find (+ k 1) acc))))
(find 1 0.0001)
\end{lstlisting}
$\rightarrow$ 10

\medskip \noindent
b.  Here's a recursive version of \lstinline!cont-frac!
\begin{lstlisting}
(define (cont-frac n d k)
  (define (iter i)
    (if (> i k)
	0
	(/ (n i)
	   (+ (d i)
	      (iter (+ i 1))))))
  (iter 1))
\end{lstlisting}
\end{document}
