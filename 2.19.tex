\documentclass[a4paper,12pt]{article}
\usepackage{listings}
\lstset{language=Lisp}

\begin{document}
\begin{lstlisting}
(define (cc amount coin-values)
  (cond ((= amount 0) 1)
        ((or (< amount 0) (no-more? coin-values)) 0)
        (else
         (+ (cc amount
                (except-first-denomination
                 coin-values))
            (cc (- amount
                   (first-denomination
                    coin-values))
                coin-values)))))

(define no-more? null?)
(define except-first-denomination cdr)
(define first-denomination car)
\end{lstlisting}
The order of the list \lstinline!coin-values! doesn't affect the
answer produced by \lstinline!cc!.  Let's proove this by induction
according to the number $m$ of the elements of
\lstinline!coin-values!.

The result is trivial for $m = 1$.  Suppose we have the result for
$m$.  Note $a_1, a_2, \ldots, a_{m+1}$ the elements of
\lstinline!coin-values!.  And let $k$ be an integer verifying
$ 2 \le k \le m+1$.  Let's show that we can swap $a_1$ and $a_k$
without modifying the value returned by \lstinline!cc!.

Note $n$ the amount we want to count the number of changes. Suppose
$n > 0$ otherwise the result is straightforward.  We have,
\begin{eqnarray*}
  \mathrm{cc}(n, a_1,\ldots,a_k,\ldots,a_{m+1}) &=&
  \mathrm{cc}(n, a_2,\ldots,a_k,\ldots,a_{m+1}) + \\ &&
  \mathrm{cc}(n-a_1,a_2,\ldots,a_k,\ldots,a_{m+1}) 
\end{eqnarray*}
Note that we have only $m$ coins left in the right-hand side of the
equality, so by induction we have
\begin{eqnarray*}
  \mathrm{cc}(n, a_1,\ldots,a_k,\ldots,a_{m+1}) &=&
  \mathrm{cc}(n, a_k, a_2,\ldots,a_{k-1},a_{k+1},\ldots,a_{m+1}) +
  \\&&
  \mathrm{cc}(n-a_1, a_k, a_2, \ldots, a_{k-1}, a_{k+1}, \ldots,
  a_{m+1}) \\ &=&
  \mathrm{cc}(n, a_2, \ldots, a_{k-1},a_{k+1},\ldots, a_{m+1}) + \\ &&
  \mathrm{cc}(n-a_k, a_2, \ldots, a_{k-1}, a_{k+1},\ldots,a_{m+1}) +
  \\ &&
  \mathrm{cc}(n-a_1, a_2, \ldots, a_{k-1}, a_{k+1}, \ldots, a_{m+1}) +
  \\ &&
  \mathrm{cc}(n-a_1-a_k, a_2, \ldots, a_{k-1}, a_{k+1}, \ldots,
  a_{m+1}) \\
\end{eqnarray*}
Grouping the first and third terms and the second and last one we
obtain
\newpage
\begin{eqnarray*}
  \mathrm{cc}(n, a_1,\ldots,a_k,\ldots,a_{m+1}) &=&
  \mathrm{cc}(n, a_1, a_2, \ldots, a_{k-1}, a_{k+1}, \ldots, a_{m+1}) +
  \\ &&
  \mathrm{cc}(n-a_k, a_1, a_2, \ldots, a_{k-1}, a_{k+1}, \ldots,
  a_{m+1}) \\ &=&
  \mathrm{cc}(n, a_k, a_1, a_2, \ldots, a_{k-1}, a_{k+1}, \ldots, a_{m+1})
\end{eqnarray*}
So finally we deduce that the order of the list doesn't affect the
result of \lstinline!cc!.
\end{document}
