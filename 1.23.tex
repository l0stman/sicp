\documentclass[a4paper,12pt]{article}
\usepackage{listings}
\newcommand{\ra}{\rightarrow}
\lstset{language=Lisp}

\begin{document}
\noindent
Let's modify \lstinline!timed-prime-test! to print only the running time
of \lstinline!prime?!.
\begin{lstlisting}
(define (timed-prime-test n)
  ((lambda (start-time)
     (prime? n)
     (- (current-inexact-milliseconds) start-time))
   (current-inexact-mill-seconds)))
\end{lstlisting}
The function \lstinline!times-for-prime! takes as paramaters a prime
number and an integer $n$ and return a list of $n$ times by applying
\lstinline!timed-prime-test! to the prime $n$ times.
\begin{lstlisting}
(define (times-for-prime p n)
  (define (loop n result)
    (if (= n 0)
	result
	(loop (- n 1)
	      (cons (timed-prime-test p)
		    result))))
  (loop n null))
\end{lstlisting}
\lstinline!average-time! returns the average of the result returned by
\lstinline!times-for-prime!.
\begin{lstlisting}
(define (average-time p n)
  ((lambda (lst)
     (/ (foldr + 0 lst)
	n))
   (times-for-prime p n)))
\end{lstlisting}
Then, we define the function \lstinline!get-prime! to return the
smallest prime number greater or equal to the parameter.
\begin{lstlisting}
(define (get-prime init)
  (define (iter n)
    (if (prime? n)
	n
	(iter (+ n 2))))
  (if (and (not (= init 2))
	   (even? init))
      (iter (+ init 1))
      (iter init)))
\end{lstlisting}
The function \lstinline!prime-list! then generates a list of $n$
primes number greater than the successive product by $10$ of $init$.
\newpage
\begin{lstlisting}
(define (prime-list init count)
  (define (iter init count result)
    (if (= count 0)
	(reverse result)
	(iter (* 10 init)
	      (- count 1)
	      (cons (get-prime init)
		    result))))
  (iter init count null))
\end{lstlisting}
And finally the function \lstinline!times-list! returns a list of the
result of \lstinline!average-time! applied to a list of prime numbers.
\begin{lstlisting}
(define (times-list lst avg-num)
  (map (lambda (p)
	 (average-time p avg-num))
       lst))
\end{lstlisting}
And finally let's define the desired number of average we compute.
\begin{lstlisting}
(define AVG-NUM 10)
\end{lstlisting}
Let's generate the list of prime numbers.
\begin{lstlisting}
(define primes (prime-list 100 12))
primes
\end{lstlisting}
$\rightarrow$ (1009
 10007
 100003
 1000003
 10000019
 100000007
 1000000007
 10000000019
 100000000003
 1000000000039
 10000000000037
 100000000000031)

Let's compute the corresponding list of times of our prime list for
the unmodified version of \lstinline!smallest-divisor!.
\begin{lstlisting}
(define before (times-list primes AVG-NUM))
\end{lstlisting}
And after we replace \lstinline!(+ test-divisor 1)! by
\lstinline!(next test-divisor)!.
\begin{lstlisting}
(define after (times-list primes AVG-NUM))
\end{lstlisting}
Finally we then obtain the list of ratios between the two versions with:
\begin{lstlisting}
(map / before after)
\end{lstlisting}
$\rightarrow$ (0.6266490765171504
 1.4490022172949
 1.4926253687315636 \\
 1.5144344533838143
 1.3868037571025473
 1.236468962089798 \\
 1.3602381543577067
 1.892754865265741
 1.8647260043702198 \\
 1.9080137170274103
 1.9071364199453982
 1.909278952905198)

We see that the ratio approach $2$ only for large values of $n$.  Wich
makes sens since our speculation deals only with order of growth.  For
small values of $n$,  the effect of the constants is not negligible.
\end{document}
