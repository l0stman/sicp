\documentclass[a4paper,12pt]{article}
\usepackage{listings}
\lstset{language=Lisp}

\begin{document}
We only modify \lstinline!adjoin-set! as follows.
\begin{lstlisting}
(define (adjoin-set x set)
  (cons x set))
\end{lstlisting}
\lstinline!element-of-set?! becomes less efficient if we allow
duplicates in the list because the two lists \lstinline!(1)! and
\lstinline!(1 1 1 1 1)! represents the same set $\{1\}$.  But if we
search for an element which is not in the list, it will take a linear
time even if there's only one element in the set.  We have the same
conclusion concerning \lstinline!intersection-set!.

On the other hand, \lstinline!adjoin-set! becomes a constant-time
operation and \lstinline!union-set! becomes linear time.

\medskip
This second implementation is more interesting for applications using
mostly union operations and seldom use intersection and to test if an
element is in the set.  But as long as the size of the entry isn't too
large, these operations won't really matters.
\end{document}
