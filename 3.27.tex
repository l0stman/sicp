\documentclass[a4paper,12pt]{article}
\usepackage{listings}
\lstset{language=Lisp}

\begin{document}
Note $T(n)$ the number of steps to compute the $n^{th}$ Fibonacci number.
If $n > 1$, we have
\[ \mathrm{fib}(n) = \mathrm{fib}(n-1) + \mathrm{fib}(n-2).\]
Thus, to compute $\mathrm{fib}(n)$, we need to compute
$\mathrm{fib}(n-2)$.  But once we have computed $\mathrm{fib}(n-2)$,
we have in the table the values of $\mathrm{fib}(k)$ for $0 \le k\le
n-2$.  We then have,
\[ T(n) = T(n-2) + \Theta(1)\]
So finally $T(n) = \Theta(n)$.

\medskip
If we have defined \lstinline!memo-fib! as \lstinline!(memoize fib)!,
the number of steps is again exponential because no value in the
computation tree is cached.  Moreover given that the table is local,
no subsequent call to \lstinline!memo-fib! could references the
already computed values.
\end{document}
