\documentclass[a4paper,12pt]{article}
\usepackage{listings}
\newcommand{\subpar}[1] {\medskip \noindent #1.}
\lstset{language=Lisp}

\begin{document}
For generic operations with explicit dispatch, each time a new data
type, we need to update all the dispatches.  The problem for explicit
dispatch too is that each implementation of an operation for each data
should know the other implementations to avoid name conflicts and to
know which operator does what with a particular kind of data.

In data-directed style, adding a new generic operation doesn't touch
the data.  But adding a new data type needs to update all the generic
operations.  Each operation implementation in for a data type doesn't
care about the implementations by the other data types.

In a message passing style, each new operation needs to be implemented
by each data type.  A new data type needs only to implement all the
operations without touching the generic operations.

If new types must often be added,  message-passing is the most
appropriate.  On the other hand, if new operations must often be added
we should choose a data-directed style.
\end{document}
