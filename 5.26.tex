\documentclass[a4paper,12pt]{article}
\usepackage{listings}
\lstset{language=Lisp}
\newcommand{\subpar}[1]{\medskip \noindent #1.}

\begin{document}

We obtain the following simulation

\begin{lstlisting}
;;; EC-EVAL input:
(define (factorial n)
  (define (iter product counter)
    (if (> counter n)
        product
        (iter (* counter product)
              (+ counter 1))))
  (iter 1 1))
(total-pushes = 3 maximum-depth = 3)

;;; EC-Eval value:
ok

;;; EC-EVAL input:
(factorial 1)
(total-pushes = 64 maximum-depth = 10)

;;; EC-Eval value:
1

;;; EC-EVAL input:
(factorial 2)
(total-pushes = 99 maximum-depth = 10)

;;; EC-Eval value:
2

;;; EC-EVAL input:
(factorial 3)
(total-pushes = 134 maximum-depth = 10)

;;; EC-Eval value:
6

;;; EC-EVAL input:
(factorial 4)
(total-pushes = 169 maximum-depth = 10)

;;; EC-Eval value:
24

;;; EC-EVAL input:
(factorial 5)
(total-pushes = 204 maximum-depth = 10)

;;; EC-Eval value:
120

;;; EC-EVAL input:
(factorial 6)
(total-pushes = 239 maximum-depth = 10)

;;; EC-Eval value:
720
\end{lstlisting}

\subpar{a} Hence, the maximum depth required to evaluate $n!$ is $10$.

\subpar{b} And from the same numbers, we deduce that the total number
of push operations used in evaluating $n!$ is $35 n + 29$.

\end{document}
